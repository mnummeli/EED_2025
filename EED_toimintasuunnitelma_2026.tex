\documentclass[a4paper,12pt]{article}
\usepackage[finnish]{babel}
\usepackage[utf8]{inputenc}
\usepackage[T1]{fontenc}
\usepackage{lmodern}
\usepackage{eurosym}
\usepackage[pdftex,colorlinks]{hyperref}
\title{Etelä-Espoon sosialidemokraattinen työväenyhdistys - toimintasuunnitelma vuodelle 2026}
\author{EED}
\date{15.09.2025}
\begin{document}
\maketitle
\section{Yleistä}
Etelä-Espoon sosialidemokraattinen työväenyhdistys r.y. on oman alueensa vahva vaikuttaja, joka haluaa välittää asukkaiden äänen Espoon kaupungin päätöksentekoon. Osastomme kuuluu jäsenenä Suomen sosialidemokraattiseen puolueeseen. Tehtävämme on edistää puolueosastona sosialidemokratian tavoitteiden ja Espoon sosialidemokraattisen kunnallisjärjestön vuodelle 2026 tehdyn toimintasuunnitelman toteutumista.
\section{Tavoitteet vuodelle 2026}
\subsection{Jäsenistö, koulutus ja sivistys}
Vuoden 2024 yhdistykselle asetetut tavoitteet jäsenten osalta ovat:
\begin{itemize}
\item{Jäsenmäärän kasvu, painottuen nuoriin ja työikäisiin}
\item{Jäsenistön koulutus verkkoviestinnän hallinnassa}
\item{Eräiden jäsentemme osallistuminen Työväen Sivistysliiton Espoon ja Kauniaisten opintojärjestö r.y:n koulutusten järjesteämiseen, 1-2 jäsenemme hallituspaikka opintojärjestössä}
\end{itemize}
\subsection{Talous}
Yhdistys aikoo jatkaa huolellista taloudenpitoa omien sijoitustensa ja Falklammen kiinteistön suhteen. Falklammen vuokratoiminta on jäsenpainoitteista, jonka tavoitteena on suunnilleen kattaa vuokra-ansioilla kiinteistön ylläpitokulut. Jäsenille myönnetään vuokrasta johtokunnan päätösten mukainen jäsenalennus. Yhdistyksen sijoitustoiminta on rahastosijoituspainotteista ja muusta asunto- tai kiinteistösijoittamisesta Falklammen lisäksi on luovuttu, koska niihin liittyy riskejä ja vastaavien johtokunnan jäsentemme ylimääräisen vaativan työn tarvetta.
\subsection{Tapahtumien järjestäminen}
Yhdistys osallistuu Espoon kunnallisjärjestön tapahtumiin, esimerkiksi toreille, ruusujen jakoon, laskiais- ja ystävänpäivätapahtumiin ja vapputapahtumiin. Näissä pyritään myös mainostamaan yhdistystä jäsenhankinnan kannalta.
\subsection{Yhteisen matkan toteuttaminen}
Yhdistys toteuttaa yhteisen matkan joko lähialueille tai kauemmas Keski- tai Etelä-Eurooppaan. Matkaa ei ole tarkoitus toteuttaa Euroopan Unionin tai Euroopan talousalueen ulkopuolisiin maihin. Matkalle osallistuville yhdistyksen jäsenille yhdistys maksaa matkakuluista osan johtokunnan päätösten mukaan, ei kuitenkaan yli 50\%, paitsi, jos kyse on tarkkaan harkituista poikkeuksista erityisryhmiin kuuluvien jäsenten tai kunnia/vapaajäsenten kohdalla.
\subsection{Poliittiset kannanotot}
Yhdistys tekee poliittisia kannanottoja, jotka ovat sosialidemokraattisten arvojen mukaisia ja huomioivat työntekijöiden oikeudet, yhdenvertaisuuden, oikeuden riittävään sosiaaliturvaan, asumiseen sekä koulutukseen ja sivistykseen. Kannanottoja voidaan tehdä yhdessä kunnallisjärjestön ja piirin kanssa ja huomioida niissä myös nykyisen oikeistohallituksen politiikan seuraukset.
\subsection{Luottamustehtävissä vaikuttaminen}
Yhdistyksen kaupunginvaltuutetut ja varavaltuutetut edistävät Espoon kaupunginvaltuustossa politiikkaa, joka toimii Espoon asukkaiden hyödyksi, huomioi kaavoituksessa kaupungin väestön kasvun luomat haasteet, liikenneyhteyksien toimivuuden ja luontoarvot. Sama koskee lautakunnissa vaikuttavia jäseniämme lautakuntansa toiminta-alueen osalta sekä kunnallisjärjestön hallituksessa, edustajistossa, piirihallituksessa ja piiriedustajistossa vaikuttavia jäseniämme. Lisäksi yhdistys kannattaa politiikkaa, joka takaa kouluihin, varhaiskasvatukseen ja vanhuspalveluihin riittävät resurssit.
\end{document}
