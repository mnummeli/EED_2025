\documentclass[a4paper,12pt]{article}
\usepackage[finnish]{babel}
\usepackage[utf8]{inputenc}
\usepackage[T1]{fontenc}
\usepackage{lmodern}
\usepackage{eurosym}
\usepackage[pdftex,colorlinks]{hyperref}
\title{Etelä-Espoon sosialidemokraattinen työväenyhdistys - toimintakertomus vuodelta 2024}
\author{EED}
\date{12.03.2025}
\begin{document}
\maketitle
\section{Yleistä}
Vuonna 2024 pidettiin europarlamenttivaalit, joissa yhdistyksemme jäsen Maria Guzenina tuli valituksi europarlamentaarikoksi. Samalla kertaa yhdistyksemme jäsen Miapetra Kumpula-Natri palasi eduskuntaan kansanedustajaksi. Yhdistyksen johtokunta kokoontui säännöllisesti.
\section{Vuoden 2024 tavoitteiden toteutuminen}
Yhdistyksen tavoitteet europarlamenttivaalien osalta toteutuivat edellä olevan kappaleen nojalla. Puolueemme presidenttiehdokas Jutta Urpilainen ei kuitenkaan tullut valituksi Tasavallan presidentiksi. Jäsenmäärässä ei tapahtunut merkittävää kasvua. Verkkoviestintää ei saatu tehostettua tarpeeksi ja sivustomme uudistaminen on kesken. Yhteinen matka Kyprokselle toteutui. Yhdistyksen jäsenet toimivat aktiivisesti Työväen Sivistysliiton Espoon ja Kauniaisten opintojärjestön hallituksessa.
\section{Talous}
Yhdistyksen taloudellinen tilanne pysyi hyvänä koko vuoden ja merkittävä osa omaisuudesta on sijoitusrahastoissa sekä Falklammen kiinteistössä. Suunnitellun mukaan yhdistys luopui tavoitteestaan hankkia uutta sijoitusasuntoa k.o. omaisuuden työlään ylläpidon vuoksi.
\section{Tapahtumien järjestäminen}
Yhdistys osallistui eurovaalien ja presidentinvaalien kampanjointiin, järjesti laskiaistapahtuman ja ruusujen jakoa kauppakeskuksissa. Lisäksi joulun alla järjestettiin puurojuhla.
\section{Poliittiset kannanotot}
Yhdistys ei tehnyt omissa nimissään merkittäviä poliittisia kannanottoja hallituksen toimintaan. Puolue on kuitenkin ollut asiassa hyvin aktiivinen.
\section{Luottamustehtävät}
Yhdistyksemme jäsenillä oli seuraavia merkittäviä luottamustehtäviä vuonna 2024 valtiollisissa hallintoelimissä, Espoon kaupungilla, Uudenmaan hyvinvointialueella ja puolueemme järjestöissä. HUOM: Osa tehtävistä on muuttunut syyskokouksen 2024 valintojen vuoksi tälle vuodelle!
\begin{itemize}
\item{Europarlamentti:} Maria Guzenina
\item{Eduskunta:} Miapetra Kumpula-Natri
\item{Aluevaltuusto:} Markku Sistonen
\item{Kaupunginhallitus:} Markku Sistonen
\item{Kaupunginvaltuusto:} Markku Sistonen
\item{Lautakunnat:}
\item{Uudenmaan piirijärjestön edustajisto:} Martti Hellström, Mikko Nummelin (varalla), Helena Tiitinen, Anja Jaatinen (varalla)
\item{Kunnallisjärjestön edustajisto:} Anja Jaatinen, Mikko Nummelin, Anita Paatelma, Markku Sistonen, Martti Hellström (varalla), Jonne Hänninen (varalla), Matti Ikonen (varalla), Helena Tiitinen (varalla)
\item{Kunnallisjärjestön hallitus:} Mikko Nummelin, Markku Helén (seuraajajäsen)
\item{Naispiirijärjestön edustajat:} Anja Jaatinen, Liisa Kivimäki, Anita Paatelma, Anneli Rönnberg, Helena Tiitinen
\end{itemize}
\end{document}
