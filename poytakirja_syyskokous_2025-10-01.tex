\documentclass[a4paper,12pt]{article}
\usepackage[finnish]{babel}
\usepackage[utf8]{inputenc}
\usepackage[T1]{fontenc}
\usepackage{lmodern}
\usepackage{eurosym}
\usepackage[pdftex,colorlinks]{hyperref}
\renewcommand{\thesubsection}{\arabic{subsection}}
\title{Etelä-Espoon sosialidemokraattinen työväenyhdistys r.y.}
\author{EED}
\date{01.10.2025}
\begin{document}
\maketitle
\tableofcontents
\section*{Sääntömääräinen syyskokous}
\subsection*{Paikka ja aika}
Hotelli Matts, keskiviikkona 01.09.2025 kello 18:30.
\subsection*{Läsnä}
Markku Sistonen \\
Mikko Nummelin \\
Anita Paatelma \\
Anja Jaatinen \\
Saila Harju \\
Ritva Sievinen \\
Tuomas Ortia \\
Aliisa Harju \\
Saini Ortia \\
Raimo Helén \\
Martti Hellström \\
Niina Hänninen \\
Janne Koski \\
Liisa Kivimäki \\
Sinikka Suontio \\
Kaija Viitakoski \\
Esko Viitakoski \\
Irja Ikonen \\
Matti Ikonen \\
Arja Kivimäki \\
Pauliina Sistonen \\
Jukka Paukku \\
Juhani Luhtanen \\
Leena Luhtanen
\section*{Pöytäkirja}
\subsection{Kokouksen avaus}
Puheenjohtaja avasi kokouksen kello 18:42.
\subsection{Kokouksen laillisuus ja päätösvaltaisuus}
Kokous todettiin lailliseksi ja päätösvaltaiseksi. Anita Paatelma esitti moitteen, jonka mukaan sääntömääräistä syyskokousta ei kutsuttu koolle hyvän tavan mukaisesti, koska kutsua ei oltu esitetty jäsenkirjeessä vaan ainoastaan Länsiväylän ilmoituksessa. Esitetään, että jatkossa puolueosaston kevät- ja syyskokouskutsut lähetetään aina myös kirjeitse ja sähköpostina. Jäsenistön on saatava tieto puolueosaston kokouksista tasavertaisesti. Syksyn 2025 puolueosaston kutsutapa ei täytä tätä vaatimusta.
\subsection{Kokouksen puheenjohtajan valinta}
Kokouksen puheenjohtajaksi valittiin Markku Sistonen (15 ääntä) vs. Raimo Helén (8 ääntä).
\subsection{Kokouksen sihteerin valinta}
Kokouksen sihteeriksi valittiin .
\subsection{Kokouksen pöytäkirjantarkastajien ja ääntenlaskijoiden valinta}
Kokouksen pöytäkirjantarkastajaksi valittiin Liisa Kivimäki ja Niina Hänninen. Sovittiin, että samat henkilöt toimivat tarvittaessa myös ääntenlaskijoina.
\subsection{Yhdistyksen puheenjohtajan valinta}
Yhdistyksen puheenjohtajaksi valittiin 2 vuodeksi Markku Sistonen. Yhdistyksen puheenjohtaja toimii samalla myös Falklampi Oy:n hallituksen puheenjohtajana.
\subsection{Yhdistyksen varapuheenjohtajan valinta}
Yhdistyksen varapuheenjohtajaksi valittiin 2 vuodeksi Sinikka Suontio. Yhdistyksen varapuheenjohtaja toimii samalla myös Falkampi Oy:n hallituksen varapuheenjohtajana.
\subsection{Yhdistyksen johtokunnan muiden jäsenten valinta}
Yhdistyksen johtokunnan jäseniksi valittiin 2 vuodeksi Arja Kivimäki, Pauliina Sistonen, Matti Ikonen, Martti Hellström, Mikko Nummelin ja Jonne Hänninen. Yhdistyksen johtokunta toimii samalla myös Falklampi oy:n hallituksena.
\subsection{Yhdistyksen toimintasuunnitelman 2026 hyväksyminen}
Yhdistyksen toimintasuunnitelma vuodelle 2026 hyväksyttiin 13 (johtokunnan pohjaesitys)-9 (Anita Paatelman muutosesitys). Muutosesitys oli seuraavanlainen:
\begin{quote}
Johtokunnan pohja: \\
Yhdistys jatkaa huolellista taloudenpitoa. Falklammen kiinteistön
vuokratuotoilla varaudutaan tuleviin korjauksiin. Yhdistyksen
jäsenille vuokrasta myönnetään johtokunnan päätösten mukainen
jäsenalennus.

Anita Paatelma: \\
Falklammen vuokrauksesta tiedotetaan
jäsenistöä heti alkuvuodesta. Vuokraus tapahtuu tasapuolisesti
jäsenistön toiveita kuunnellen. Mökin vuokraus perustuu
omatoimisuuteen, kiinteistöstä pidetään huolta lähinnä talkoovoimin.
Talkoot ja muut mahdolliset mökkitapahtumat kasvattavat
yhteisöllisyyttämme. Tavoittelemme tilannetta, jossa vuokratulot
kattavat pakolliset menot. Investoinnit tuodaan johtokunnan
päätettäväksi. Hintataso pidetään kohtuullisena, jotta kaikilla
jäsenillä olisi mahdollisuus nauttia loman vietosta yhdistyksen
lomapaikassa.
\end{quote}
\subsection{Yhdistyksen talousarvion 2026 hyväksyminen}
Yhdistyksen talousarvion vuodelle 2026 hyväksyttiin.
\subsection{Falklampi Oy:n talousarvion 2026 hyväksyminen}
Falklampi Oy:n talousarvion vuodelle 2026 hyväksyttiin.
\subsection{Yhdistyksen tili- ja toiminnantarkastajien valinta}
Yhdistyksen toiminnantarkastajaksi valittiin Janne Koski äänin 13-9 (Vastaehdotus: Maria Rajala / es. Paatelma)
\subsection{Kunnallisten luottamushenkilöiden valinta}
Espoon sosialidemokraattisen kunnallisjärjestön edustajiksi valittiin Markku Sistonen, Pauliina Sistonen, Sinikka Suontio ja Raimo Helén. Espoon sosialidemokraattisen kunnallisjärjestön varaedustajiksi valittiin  Mikko Nummelin, Liisa Kivimäki, Niina Hänninen ja Janne Koski. Espoon sosialidemokraattisen kunnallisjärjestön hallitukseen valittiin jäseneksi Mikko Nummelin/- ja seuraajajäseneksi (2 kpl) . Uudenmaan sosialidemokraattisen piirijärjestön edustajiston jäseniksi varajäsenineen valittiin Helena Tiitinen/Sinikka Suontio ja (Jonne Hänninen)/Mikko Nummelin. Uudenmaan sosialidemokraattisen naispiirijärjestön jäseniksi valittiin Helena Tiitinen, Anja Jaatinen ja Anita Paatelma.
\subsection{Kokouksen päättäminen}
Puheenjohtaja päätti kokouksen kello 21:05.
\section*{Allekirjoitukset}
\begin{flushleft}
\begin{tabular}{ll}
& \\
& \\
& \\
Markku Sistonen, puheenjohtaja &
Mikko Nummelin, sihteeri \\
& \\
& \\
& \\
Liisa Kivimäki &
Niina Hänninen
\end{tabular}
\end{flushleft}
\end{document}
