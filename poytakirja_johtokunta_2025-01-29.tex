\documentclass[a4paper,12pt]{article}
\usepackage[finnish]{babel}
\usepackage[utf8]{inputenc}
\usepackage[T1]{fontenc}
\usepackage{lmodern}
\usepackage{eurosym}
\usepackage[pdftex,colorlinks]{hyperref}
\renewcommand{\thesubsection}{\arabic{subsection}}
\title{Etelä-Espoon sosialidemokraattinen työväenyhdistys r.y.}
\author{EED}
\date{29.01.2025}
\newtheorem{aanestys}{Äänestys}
\begin{document}
\maketitle
\tableofcontents
\section*{Johtokunnan kokous}
\subsection*{Läsnä}
\begin{flushleft}
  Markku Sistonen, puheenjohtaja \\
  Anita Paatelma, varapuheenjohtaja \\
  Martti Hellström \\
  % Jonne Hänninen \\
  Matti Ikonen \\
  Arja Kivimäki \\
  Mikko Nummelin \\
  Pauliina Sistonen \\
  Helena Tiitinen \\
  Raimo Helén, 1. varajäsen, äänioikeus \\
  Saini Ortia, 2. varajäsen, puheoikeus \\
  Anja Jaatinen, sihteeri
\end{flushleft}
\subsection*{Paikka ja aika}
Ison Omenan kirjastossa, 29.01.2025 kello 18:00.
\section*{Pöytäkirja}
\subsection{Kokouksen avaus}
Puheenjohtaja avasi kokouksen kello 18:00.
\subsection{Kokouksen laillisuus ja päätösvaltaisuus}
Kokous todettiin lailliseksi ja päätösvaltaiseksi.
\subsection{Kokouksen pöytäkirjantarkastajien ja ääntenlaskijoiden valinta}
Pöytäkirjantarkastajiksi ja ääntenlaskijoiksi valittiin Arja Kivimäki ja Mikko Nummelin.
\subsection{Johtokunnan järjestäytyminen}
Johtokunnan sihteeriksi valittiin Mikko Nummelin. Johtokunnan rahastonhoitajaksi ja jäsenasiainhoitajaksi valittiin Arja Kivimäki. Tiedotusvastaaviksi valittiin Markku Sistonen, Martti Hellström ja Helena Tiitinen. Verkkosivuvastaaviksi valittiin Martti Hellström ja Mikko Nummelin.
\begin{aanestys}[Johtokunnan sihteerin valinta]
  Mikko Nummelin vs. Anja Jaatinen, äänet: 6-4.
\end{aanestys}
\subsection{Yhdistyksen ja Falklammen taloustilanne}
Arja Kivimäki esitteli yhdistyksen ja Falklammen taloustilanteen. Nämä oli jaettu etukäteen johtokunnan jäsenille sähköpostitse.
\subsection{Jäsenasiat}
Yhdistykseen hyväksyttiin yksi uusi jäsen.
\subsection{Yhdistyksen nimenkirjoitus- ja käyttöoikeudet}
Yhdistyksen nimenkirjoitus- ja käyttöoikeudet päätettiin antaa Markku Sistoselle ja Arja Kivimäelle. Valinnasta lähetetään pöytäkirjanote Osuuspankkiin, Patentti- ja rekisterihallitukselle ja Verohallinnolle ja lisätään merkintä jäsenrekisteriin.
\subsection{Kevätkokouksen ajankohdasta päättäminen}
Kevätkokouksen ajankohdaksi päätettiin keskiviikko 12.3. kello 18:30.
\subsection{Kevään tapahtumat}
\begin{itemize}
\item{Johtokunnan kokous} (Ison Omenan kirjastossa 29.1.2025 kello 18)
\item{Naistenpäivän tapahtuma} (Lippulaivassa 8.3.2025)
\item{Johtokunnan kokous} (keskiviikkona 12.3.2025 kello 17:30)
\item{Kevätkokous} (keskiviikkona 12.3.2025 kello 18:30)
\item{Kunta- ja aluevaalit} (sunnuntaina 13.4.2025)
\item{Wappu} (ke 30.4. - to 1.5.2025)
\end{itemize}
\subsection{Muut asiat}
Yhdistys on sopinut Työväen sivistysliiton Espoon ja Kauniaisten opintojärjestö r.y:n (jälk. Opintojärjestö) kanssa, että Yhdistys saa käyttää opintojärjestön verkkosivuja ehdokkaidensa vaalimainontaan. Opintojärjestö järjestää Yhdistyksen ehdokkaista kuvakollaasin ja tarjoaa kullekin ehdokkaalle tunnuksen vaalimainosten ja blogikirjoitusten julkaisemiseksi Opintojärjestön sivulle. Vastineeksi Yhdistys suorittaa Opintojärjestölle maksun, joka kattaa Opintojärjestölle Kapsin verkkopalvelulle ja Louhen nimipalvelulle aiheutuvat kulut laskua vastaan.
\subsection{Kokouksen päättäminen}
Puheenjohtaja päätti kokouksen 19:30.
\section*{Allekirjoitukset}
\begin{flushleft}
\begin{tabular}{ll}
& \\
& \\
& \\
Markku Sistonen, puheenjohtaja &
Anja Jaatinen, sihteeri \\
& \\
& \\
& \\
Arja Kivimäki &
Mikko Nummelin
\end{tabular}
\end{flushleft}
\end{document}
